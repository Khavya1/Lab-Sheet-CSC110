\documentclass[11pt]{report}
\usepackage[utf8]{inputenc}
\usepackage{fancyhdr}
\usepackage{color}
\usepackage{listings}
\usepackage{amsmath}
\usepackage{geometry}
\geometry{
 a4paper,
 right=14mm,
 bottom=20mm,
 top=18mm,
}

\lstset{
    frame=single,
    breaklines=true,
    postbreak=\raisebox{0pt}[0pt][0pt]{\ensuremath{\hookrightarrow\space}},
    basicstyle=\ttfamily,
    showstringspaces=false,
}
 
\renewcommand{\chaptername}{Program}

\title{Lab Sheet 04 \\
      \textbf{C Programming} \\
      CSC-110}
\author{\textbf{Anukul Adhikari} \\ 11/23/2076}
\date{\today}

\begin{document}
\maketitle

\chapter*{Information}
All code used in this report including source code of this report itself is available at:\\ \texttt{https://github.com/beinganukul/Lab-Sheet-CSC110}
\section*{Software}
Following compiler and configuration is verified to work with the snippets in this report:\\
\texttt{Compiler - gcc version 9.2.1 20200130\\
Compiler target - Arch Linux 9.2.1+20200130-2
}

\tableofcontents

\chapter{Simple Calculator}
\section{Problem Statement}
Write a program that takes input of two numbers and performs addition,subtraction,multiplication,division between them using user defined function called $add$(),$sub$(),$div$() and $mul$().
\section{Program}
\lstinputlisting[language=c]{01-calculator-function.c}

\chapter{Which is Largest one?}
\section{Problem Statement}
Write a program to find out the largest among threee numbers using user defined function.
\section{Algorithm}
\begin{itemize}
  \item Accept input and save the integer as \texttt{number}.
  \item Check if $a$ is greater than $b$.
  \subitem -If true check if $a$ is greater than $c$
  \subitem -If true a is greatest 
  \subsubitem -If false $c$ is greatest
  \subitem -If flase check if $b$ is greater than $c$.
  \subitem -If true $b$ is greatest 
  \subsubitem -If false $c$ is greatest
\end{itemize}
\leavevmode
\section{Program}
\lstinputlisting[language=c]{02-compare.c}

\chapter{Power Function}
\section{Multiplication}
Write a function which recieves a float and int from main(),finds the product of these two and returns the product which is printed through main.
\section{Compilation Note}
You must define function prototype to propperly compile the code otherwise $gcc$ compiler will throw $Conflicting  Types  for  Error$ - Is a common mistake in programming it occurs due to incompatibility of parameters/arguments type in declaration and definition of the function.
\section{Program}
		\lstinputlisting[language=c]{03-multiply.c}

\chapter{Prime Numbers}
\section{Problem Statement}
Write a program to check if the given number is prime or not using user defined function. 
\section{Algorithm}
\begin{itemize}
  \item Accept input and save the integer as \texttt{a}.
  \item A for loop is iterated from $i = 2 to i < n/2$.
  \item In each iteration, whether $n$ is perfectly divisible by $i$ is checked.
  \item If $n$ is perfectly divisible by $i$, $n$ is not a prime number. In this case, flag is set to $1$, and the loop is terminated using the break statement.
  \item After the loop, if $n$ is a prime number, flag will still be $0$.
  \item However, if $n$ is a non-prime number, flag will be $1$.
\end{itemize}
\leavevmode
\section{Program}
\lstinputlisting[language=c]{04-prime-check.c}

\chapter{Factorial Using User Defined Function}
\section{Problem Statement}
Write a program to find factorial of given number using user defined function named long int factorial($int$).

\section{Program}
\lstinputlisting[language=c]{05-factorial.c}

\chapter{Power Function}
\section{Problem Statement}
Write a program to calculate $a$ raised to power $b$ using user defined function with following prototype int power($int$,$int$).
\section{Algorithm}
To use pow function you need to define another header file math.h
NOTE: For $gcc$ users while compiling add $-lm$ to link math library else it won't work.
\section{Program}
		\lstinputlisting[language=c]{06-power.c}
		

\chapter{Recursion in Factorial}
\section{Problem Statement}
Write an algorithm and program to compute the following using recursion.
\begin{enumerate}
  \item factorial of an integer $n$.
\end{enumerate}
\section{Algorithm}
The factorial of a positive integer $n$ can be obtained recursively using the following algorithm.
\begin{itemize}
\item If $n=0$, return 1.
\item Multiply $n$ by $(n-1)!$ and return the result. 
\end{itemize}

\section{Program}
        \lstinputlisting[language=c]{07-factorial-recursion.c}

\chapter{Recursion in Sum-Series}
\section{Problem Statement}
Write an algorithm and program to compute the following using recursion.
\begin{enumerate}
  \item calculate sum of series $1+2+3+4+.....+n$ using recursion.
\end{enumerate}
\section{Algorithm}
The sum of series of a positive integer upto $n$ can be obtained recursively using the following algorithm.
\begin{itemize}
\item If $n=0$, return 1.
\item Add $n$ with $(n-1)$ and return the result. 
\end{itemize}

\section{Program}
        \lstinputlisting[language=c]{08-sum-series-recursion.c}

\chapter{Recursion in Fibonacci}
\section{Problem Statement}
Write an algorithm and program to compute the following using recursion.
\begin{enumerate}
  \item calculate $n$ number of fibonacci numbers using recursion.
\end{enumerate}
\section{Algorithm}
The sum of series of a positive integer upto $n$ can be obtained recursively using the following algorithm.
\begin{itemize}
\item If $n=0$, return 0.
\item If $n=1$, return 1.
\item Add $(n-1)$ with $(n-2)$ and return the result. 
\end{itemize}

\section{Program}
        \lstinputlisting[language=c]{09-fibonacci-using-recursion.c}

\chapter{Recursion in Power}
\section{Problem Statement}
Write an algorithm and program to compute the following using recursion.
\begin{enumerate}
  \item computation of $a^b$ (a raised to the power b)
\end{enumerate}
\section{Algorithm}
The result of computation of $a^b$ (a raised to the power b) can be obtained using following algorithm.
\begin{itemize}
\item If $b = 1$, return $a$.
\item Multiply $a$ by the result of $a^{b-1}$.
\end{itemize}
\section{Program}
        \lstinputlisting[language=c]{10-power-raised-recursion.c}
        
\chapter{Sum Of Finite Series}
\section{Problem Statement}
Write a program to calculate the value of the following finite series.
$1+x+x^2/2!+x^3/3!+..$ upto n terms
\section{Program}
\lstinputlisting[language=c]{11-finite-series.c}

\chapter{Sum of Natural Numbers}
\section{Problem Statement}
Write a program to find the sum of first twenty natural numbers using function.
\section{Program}
\lstinputlisting[language=c]{12-sum-of-natural-numbers.c}

\end{document}
